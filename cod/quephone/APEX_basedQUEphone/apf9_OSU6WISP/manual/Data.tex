%~~~~~~~~~~~~~~~~~~~~~~~~~~~~~~~~~~~~~~~~~~~~~~~~~~~~~~~~~~~~~~~~~~~~~~~~
% $Id: Data.tex,v 1.3.2.2 2008/10/01 14:48:57 dbliudnikas Exp $
%~~~~~~~~~~~~~~~~~~~~~~~~~~~~~~~~~~~~~~~~~~~~~~~~~~~~~~~~~~~~~~~~~~~~~~~~
% RCS Log:
%
% $Log: Data.tex,v $
% Revision 1.3.2.2  2008/10/01 14:48:57  dbliudnikas
% Apf9i Seascan TD: update conversion sections.
%
% Revision 1.3.2.1  2008/10/01 13:07:23  dbliudnikas
% Update for replacement of SBE41CP CTD with Seascan TD.
%
% Revision 1.3  2007/10/05 22:33:48  swift
% Change telemetry retry interval and add definitions of bit-masks for status words.
%
% Revision 1.2  2006/11/24 23:58:15  swift
% Updated format descriptions to account for changes in Sbe41cp firmware.
%
% Revision 1.1  2006/11/03 19:08:57  swift
% Added user manual to CVS control.
%
% Revision 1.2  2006/07/10 22:24:49  swift
% Modifications to bring the manual up to date with
% changes to the SeaBird CTD firmware (v1.1c).
%
% Revision 1.1  2006/04/11 15:32:32  swift
% Initial revision
%
%~~~~~~~~~~~~~~~~~~~~~~~~~~~~~~~~~~~~~~~~~~~~~~~~~~~~~~~~~~~~~~~~~~~~~~~~

\section{Telemetered data.}
\label{sec:TelemeteredData}

Iridium floats telemeter two kinds of data for each profile cycle and these
data are transferred as separate files: message files and log files.
Message files contain hydrographic data and follow a naming convention
\emph{FloatId}.\emph{ProfileId}.msg where \emph{FloatId} is the 4-digit
serial number of the float controller and \emph{ProfileId} is the 3-digit
profile counter.  Log files contain detailed engineering data with
time-stamped diagnostics of float operations.  Examples of message files and
log files from actual floats can be found in
Appendixes~\ref{sec:IridiumMsgSample} and \ref{sec:IridiumLogSample}.  It
will be helpful to refer to these examples as you read this section.

\subsection{Format specification for APF9i firmware.}

This section summarizes the format specification for hydrographic data
telemetered by Iridium floats.  The ``official'' format specification can be
found in the ``src'' directory of your distribution.  Any discrepancy
between \textbf{src/FormatNotes} and the information in this manual should
be resolved in favor of the former.

Iridium message files end with a ".msg" extension.  Each iridium message
file consists of blocks of similar data presented in the order that they
were collected during the profile cycle.  This firmware revision includes
five blocks of data:

\begin{enumerate}
\item Park-phase PT samples: These are hourly low-power PT samples collected
      during the park phase of the profile cycle.

\item Low resolution PT samples: The deep parts of the profile can be
      represented using low-resolution spot samples collected at
      predetermined pressures.  Low resolution spot sampling in the deep
      water was implemented as an energy savings measure.
      
\item GPS fixes: After the profile is executed and the float reaches the
      surface, the location of the profile is determined via GPS.

\item Biographical and engineering data: Various kinds of biographical and
      engineering data are collected at various times during the profile
      cycle.  
\end{enumerate}

Usually, only one telemetry cycle is required to upload the data to the
remote host computer.  However, sometimes the iridium connection is broken
or the quality of the connection is so poor that the float will abort the
telemetry attempt, wait a few minutes, and then try again.  Data blocks 4
and 5 will be repeated for each telemetry cycle of a given profile.

A description of the format for each of these blocks of data follows.  A
sample Iridium-message file is available in Appendix~\ref{sec:IridiumMsgSample}.


\subsubsection{Format for park-phase PT samples.}
\label{sec:PtSample}

   Hourly low-power PT samples are collected during the park phase of the
   profile cycle.  The park phase is also when active ballasting is done.
   Each sample includes the date and time of the sample, the unix epoch (ie.,
   the number of seconds since 00:00:00 on Jan 1, 1970), the mission time
   (ie., the number of seconds since the start of the current profile
   cycle), the pressure (decibars), and the temperature ($^\circ$C).  For example:

   \begin{verbatim}
              |------- date -----|   UnixEpoch   MTime       P       T
   ParkPt:    Jul 03 2006 18:37:34  1151951854   14414  988.18  7.4971
   ParkPt:    Jul 03 2006 19:37:31  1151955451   18011  992.15  7.3613
   ParkPt:    Jul 03 2006 20:37:31  1151959051   21611  998.23  7.3428
   ParkPt:    Jul 03 2006 21:37:31  1151962651   25211 1000.38  7.2806
   ParkPt:    Jul 03 2006 22:37:31  1151966251   28811 1003.01  7.2844
   \end{verbatim}


\subsubsection{Format for low resolution PTS samples.}
\label{sec:LowResPtsSample}
   
   The Seascan TD sensor that is used on iridium floats has features that enable
   subsampling of the water column (similar to the SBE41) as well as the
   ability to bin-average a continuous sampling of the water column.  For
   subsampled data, the values of pressure, temperature, and salinity are not
   encoded but are given in conventional units (decibars, $^\circ$C, PSU).  For
   example:
      
   \begin{verbatim}
   $ Discrete samples: 6
   $       p        t  
     1002.59    3.912  (Park Sample)
     1000.11    3.929  
      947.36    4.035  
      897.56    4.163  
      847.54    4.344  
      798.09    4.478  
    \end{verbatim}

\subsubsection{Format for GPS fixes.}
   
   Each telemetry cycle begins with the float attempting to acquire a GPS
   fix.  The fix includes the amount of time required to acquire the fix,
   the longitude and latitude (degrees), the date and time of the fix, and the
   number of satellites used to determine the fix.  For example:

   \begin{verbatim}
   # GPS fix obtained in 98 seconds.
   #         lon     lat mm/dd/yyyy hhmmss nsat
   Fix: -152.945  22.544 09/01/2005 104710    8
   \end{verbatim}

   Positive values of longitude, latitude represent east, north hemispheres,
   respectively. Negative values of longitude, latitude represent west,
   south hemispheres, respectively.  The date is given in month-day-year
   format and the time is given in hours-minutes-seconds format.

   If no fix was acquired then the following note is entered into the
   iridium message:

   \begin{verbatim}
   # Attempt to get GPS fix failed after 600 seconds.
   \end{verbatim}

\subsubsection{Format for biographical and engineering data.}

   These data have the format, "key"="value", as shown in the following
   examples:

   \begin{verbatim}
   ActiveBallastAdjustments=5
   AirBladderPressure=119
   AirPumpAmps=91
   AirPumpVolts=192
   BuoyancyPumpOnTime=1539
   \end{verbatim}
    
   Interpretation of these data requires detailed knowledge of firmware
   implementations and is generally beyond the scope of this manual.
   However, there are two 16-bit status-words (status, SeascanStatus) whose
   asserted bits indicate special conditions as described below.

   The definition of each bit mask for the 16-bit ``status'' word are given in
   the table below:

   \label{StatusBitMasks}
   \begin{verbatim}
   DeepPrf            0x0001  The current profile is a deep profile.
   ShallowWaterTrap   0x0002  Shallow water trap detected
   Obs25Min           0x0004  Sample time-out (25 min) expired.
   PistonFullExt      0x0008  Piston fully extended before surface-detection algorithm terminated.
   AscentTimeOut      0x0010  Ascent time-out expired.
   DownLoadCfg        0x0020  The mission.cfg file has not been downloaded yet.
   BadSeqPnt          0x0100  Invalid sequence point detected.
   SeascanPFail       0x0200  Seascan(P) exception. 
   SeascanPtFail      0x0400  Seascan(PT) exception. 
   SeascanPtsFail     0x0800  Seascan(PTS) exception.
   SeascanPUnreliable 0x1000  Seascan(P) unreliable.
   AirSysLimit        0x2000  Pneumatic inflation limits imposed; excessive energy consumption.
   WatchDogAlarm      0x4000  Wake-up by watchdog alarm.
   PrfIdOverflow      0x8000  The 8-bit profile counter overflowed.
   \end{verbatim} 


The definition of each bit mask for the 16-bit ``SeascanStatus'' word
(Pneumonic:~SEASCAN) are given in the table below:

   \label{SbeStatusBitMasks}
   \begin{verbatim}
   SeascanPedanticExceptn 0x0001 An exception was detected while parsing the p-only pedantic regex.
   SeascanPedanticFail    0x0002 The Seascan response to p-only measurement failed the pedantic regex.
   SeascanRegexFail       0x0004 The Seascan response to p-only measurement failed the nonpedantic regex.
   SeascanNullArg         0x0008 NULL argument detected during p-only measurement.
   SeascanRegExceptn      0x0010 An exception was detected while parsing the p-only nonpedantic regex.
   SeascanNoResponse      0x0020 No response detected from Seascan for p-only request.
                          0x0040 Not used yet.
                          0x0080 Not used yet.
   SeascanPedanticExceptn 0x0100 An exception was detected while parsing the pts pedantic regex.
   SeascanPedanticFail    0x0200 The Seascan response to pts measurement failed the pedantic regex.
   SeascanRegexFail       0x0400 The Seascan response to pts measurement failed the nonpedantic regex.
   SeascanNullArg         0x0800 NULL argument detected during  pts measurement.
   SeascanRegExceptn      0x1000 An exception was detected while parsing the pts nonpedantic regex.
   SeascanNoResponse      0x2000 No response detected from Seascan TD for pts request.
                          0x4000 Not used yet.
                          0x8000 Not used yet.
   \end{verbatim} 

\subsection{Engineering log files.}

The engineering log files contain time-stamped entries of what the float was
doing at any given time.  Every nook and cranny of the float firmware has
self-monitoring features built in that are a synthesis of self-adaptive and
user-controlled behaviors.  The self-adaptive nature stems from the fact
that if the float firmware detects problems or difficulties then engineering
log entries are automatically generated as an aid to on-shore diagnostics.
The user-controlled nature stems from the fact that the user can remotely
adjust the verbosity of the engineering logs using the 2-way
\textbf{Verbosity} command.  Refer to Section~\ref{sec:RemoteControl} for
information about 2-way (ie., remote) float configuration.

\section{Processed data.}
\label{sec:ProcessedData}

Decoding and processing the message files from this \iridium\ implementation
is relatively easy because the data are ASCII and mostly self-describing.

General conversion equations for voltage, current and vacuum are provided below.

Volts = (V * 0.077) + 0.486 volts

Current = (I * 4.052) - 3.606 mA

Vacuum = (V * 0.293) - 29.767 inHg
  

%%% Local Variables: 
%%% mode: latex
%%% TeX-master: "IridiumApex"
%%% End: 
