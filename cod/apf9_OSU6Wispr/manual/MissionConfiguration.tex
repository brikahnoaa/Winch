%~~~~~~~~~~~~~~~~~~~~~~~~~~~~~~~~~~~~~~~~~~~~~~~~~~~~~~~~~~~~~~~~~~~~~~~~
% $Id: MissionConfiguration.tex,v 1.5.2.2 2008/10/01 14:50:31 dbliudnikas Exp $
%~~~~~~~~~~~~~~~~~~~~~~~~~~~~~~~~~~~~~~~~~~~~~~~~~~~~~~~~~~~~~~~~~~~~~~~~
% RCS Log:
%
% $Log: MissionConfiguration.tex,v $
% Revision 1.5.2.2  2008/10/01 14:50:31  dbliudnikas
% Apf9i Seascan TD: mission list update.
%
% Revision 1.5.2.1  2008/10/01 13:07:24  dbliudnikas
% Update for replacement of SBE41CP CTD with Seascan TD.
%
% Revision 1.5  2008/07/14 17:02:19  swift
% Added compensator hyper-retraction feature to allow floats to be
% parked at mid-water pressures.
%
% Revision 1.4  2007/10/05 22:33:48  swift
% Change telemetry retry interval and add definitions of bit-masks for status words.
%
% Revision 1.3  2007/05/07 23:48:34  swift
% Added description of TimeOfDay feature including its remote control.
%
% Revision 1.2  2006/11/22 02:50:47  swift
% Added sanity check for CP activation pressure to warn against spot sampling
% in the main thermocline.
%
% Revision 1.1  2006/11/03 19:08:57  swift
% Added user manual to CVS control.
%
% Revision 1.2  2006/07/10 22:24:49  swift
% Modifications to bring the manual up to date with
% changes to the SeaBird CTD firmware (v1.1c).
%
% Revision 1.1  2006/04/14 17:30:44  swift
% Initial revision
%
%~~~~~~~~~~~~~~~~~~~~~~~~~~~~~~~~~~~~~~~~~~~~~~~~~~~~~~~~~~~~~~~~~~~~~~~~

\newcommand{\cs}{Configuration Supervisor}

\section{Mission configuration.}
\label{sec:MissionConfiguration}

The deconstruction of the profile cycle in
Section~\ref{sec:DeconstructProfileCycle} will provide the framework for
understanding how various parameter values determine the nature of the
mission. The float's mission is configured according to the following
mission parameters:

\begin{minipage}{6in}
\begin{verbatim}
      APEX version 091208  sn 0000
      User: iridium                                  
      Pwd:  0xafb3
      Pri:  ATDT0012066855555                          Mhp
      Alt:  ATDT0012066165555                          Mha
      INACTV ToD for down-time expiration. (Minutes)   Mtc
       14400 Down time. (Minutes)                      Mtd
       00660 Up time. (Minutes)                        Mtu
       00540 Ascent time-out. (Minutes)                Mta
       00360 Deep-profile descent time. (Minutes)      Mtj
       00360 Park descent time. (Minutes)              Mtk
       00360 Mission prelude. (Minutes)                Mtp
       00015 Telemetry retry interval. (Minutes)       Mhr
       00060 Host-connect time-out. (Seconds)          Mht
           0 Continuous profile activation. (Decibars) Mc
        1000 Park pressure. (Decibars)                 Mk
        2000 Deep-profile pressure. (Decibars)         Mj
         066 Park piston position. (Counts)            Mbp
         000 Compensator hyper-retraction. (Counts)    Mbh
         016 Deep-profile piston position. (Counts)    Mbj
         010 Ascent buoyancy nudge. (Counts)           Mbn
         022 Initial buoyancy nudge. (Counts)          Mbi
         001 Park-n-profile cycle length.              Mn
         124 Maximum air bladder pressure. (Counts)    Mfb
         096 OK vacuum threshold. (Counts)             Mfv
         227 Piston full extension. (Counts)           Mff
         016 Piston storage position. (Counts)         Mfs
           2 Logging verbosity. [0-5]                  D
        0002 DebugBits.                                D
        557d Mission signature (hex).
      
\end{verbatim}
\end{minipage}\\

A description of each mission parameter follows:

\begin{description}

\item[User \& Pwd]---The user-name and password used by the float to log
  into the remote host.  The display shows an encoded version of the
  password rather than the password itself.

\item[Pri \& Alt]---The AT dialstrings used by the \iridium\ LBT (ie.,
  modem) to dial the primary and alternate remote hosts.  Two remote hosts
  are needed---reliance on only one remote host is dangerous and strongly
  discouraged.

\item[TimeOfDay]---This allows the user to specify that the down-time should
  expire at a specific time of day (ToD).  For example, the ToD feature
  allows the user to schedule profiles to happen at night.

  The ToD is expressed as the number of minutes after midnight (GMT).  The
  valid range is 0-1439 minutes.  Any value outside this range will cause
  the ToD feature to be disabled.

\item[Down-time]---The total amount of time allowed for the \emph{descent}
  and \emph{park} phases of the profile cycle.  The sequence points K
  (Section~\ref{sec:ParkProfileCycle}) and D
  (Section~\ref{sec:DeepProfileCycle}) mark the end of the down-time.  The
  valid range is 1~minute to 30~days.

  \emph{Note:} If the \textbf{TimeOfDay} feature is enabled then the length
  of the whole profile cycle will turn out to be an integral number of days.
  The user should specify the down-time to be precisely 1~day less than the
  desired length of the profile cycle.  For example, if profiles are to be
  executed every 10~days then the down-time should be specified to be 9~days
  (ie., 12960~minutes).

\item[Up-time]---The total amount of time allowed for the \emph{profile} and
  \emph{telemetry} phases of the profile cycle.  Sequence points K~[C]
  (Section~\ref{sec:ParkProfileCycle}) and D~[C]
  (Section~\ref{sec:DeepProfileCycle}) mark the beginning~[end] of the
  up-time.  The valid range is 1~minute to 24~hours.

\item[Ascent time-out]---The maximum amount of time allowed for the profile
  phase to complete.  Sequence points K~[P]
  (Section~\ref{sec:ParkProfileCycle}) and D~[P]
  (Section~\ref{sec:DeepProfileCycle}) mark the beginning~[end] of the
  ascent time-out period.  The valid range is 1~minute to 10~hours.

\item[Deep-profile descent time]---The maximum amount of time allowed for
  the float to descend from the park pressure to the deep target pressure.
  Sequence points K~[D] (Section~\ref{sec:DeepProfileCycle}) mark the
  beginning~[end] of the deep-descent period.  The valid range is 0-8~hours.

\item[Park descent time]---The amount of time allowed for the float to
  descend from the surface to the park pressure before the park phase (and
  active ballasting) begins.  The valid range is 1~minute to 8~hours.

\item[Mission prelude]---The amount of time allowed after float activation
  before the float begins its first descent.  The valid range is 1~minute to
  6~hours.

\item[Telemetry retry interval]---The amount of time after initiation of a
  telemetry attempt before initiating the next attempt (ie,. if the former
  fails). The valid range is 1~minute to 6~hours.

\item[Host-connection time-out]---The maximum amount of time allowed (after
  sending the AT dialstring) to receive the ``CONNECT'' response from the
  remote modem.  The valid range is 30~seconds to 5~minutes.

\item[Continuous profile activation]---The target pressure for activating
  the continuous profile.  During the profile phase, the firmware will stop
  collecting spot samples and initiate continuous profiling as soon as
  the float detects a pressure less than the target pressure.  Any finite
  value is valid.

\item[Park pressure]---The target pressure for the active ballasting
  mechanism.  The float firmware will seek to maintain the float at this
  pressure during the park phase.  The valid range is 0-2000~decibars.
  \textbf{Warning:} Floats with \NComp\ are buoyancy unstable in the range
  of pressures \mbox{$\sim$400--700~decibars} and must be parked outside
  this range.  Refer to Section~\ref{DescentPhaseNComp} for more details.


\item[Deep-profile pressure]---The target pressure for a deep profile.
  During the deep-descent phase, the pressure is monitored a 5~minute
  intervals.  The profile phase is initiated when the float detects a
  pressure greater than this target pressure.  The valid range is
  0-2000~decibars.

\item[Park piston position]---An initialization value for the piston
  position at the park pressure.  The autoballasting mechanism will
  automatically adjust this value to drive the float to the park pressure.
  The valid range is 1-254~counts.
\item[Compensator hyper-retraction]---This parameter exists solely for
  floats with \NComp\ and causes the piston to be temporarily retracted
  beyond the park position.  Near the end of the descent phase, the piston
  is extended back to the park position.  This implements the ability to
  park such floats at any pressure deeper than 850~decibars.  Refer to
  Section~\ref{DescentPhaseNComp} for more details. The valid range is
  0-254~counts.  Floats without \NComp\ should have this parameter set to
  zero.


\item[Deep-profile piston position]---An initialization value for the piston
  position at the deep-profile pressure.  The \apf\ firmware will
  automatically adjust this value to drive the float to the deep-profile
  pressure in the time allowed.  The valid range is 1-254~counts.

\item[Ascent buoyancy nudge]---The amount that the piston is extended when
  the ascent-control algorithm determines that more buoyancy is needed to
  maintain the minimum ascent rate of 8~millibars/second.  The valid range
  is 1-254~counts.

\item[Initial buoyancy nudge]---The amount that the piston is extended at
  the beginning of the profile phase to get the float to start ascending.
  This same extension is also applied when the surface detection algorithm
  terminates.  The valid range is 1-254~counts.

\item[Park-n-profile cycle length]---This parameter determines how often
  deep profiles should be executed.  For example, if this value is 4 then
  profiles 4, 8, 12, and so on will be deep profiles.  The valid range is
  1-254.  The value 254 is a special sentinel value that disables the
  park-n-profile feature.

\item[Maximum air bladder pressure]---This parameter determines the
  cut-off pressure when inflating the air bladder.   The valid range is
  1-240~counts. 

\item[OK vacuum threshold]---This parameter determines the threshold
  internal pressure during the float's self-test at the beginning of the
  mission.  If the internal pressure exceeds this threshold then the
  self-test will fail and the mission will be aborted.  After the mission
  starts, this value is never used again.  The valid range is 1-254~counts.

\item[Piston full extension]---This parameter determines the maximum piston
  extension allowed to prevent the buoyancy pump from self-destructing.  The
  valid range is 1-254~counts.

\item[Piston storage position]---This parameter determines the preferred
  piston extension during storage and shipment.  The valid range is
  1-254~counts.

\item[Logging verbosity]---An integer in the range [0,5] that determines the
  logging verbosity with higher values producing more verbose logging.  A
  verbosity of 2 yields standard logging.

\end{description}

The values of all mission parameters can be set via the firmware's
command-mode user interface.  In addition, a subset of these parameters can
be set via the remote control user interface (see
Section~\ref{sec:RemoteControl}).  

\subsection{The \emph{Configuration Supervisor}.}
\label{sec:ConfigSupervisor}

The objective of the \cs\ is to guard against various common classes of
misconfigurations.  When examining a given mission configuration, the \cs\
applies $\sim40$ tests that seek to detect parameters or interactions
between parameters that could be harmful or fatal to a deployed float.

However, \textbf{the \cs\ is not a substitute for a thinking human
  brain}---misconfigurations exist that can not be detected by firmware but
which are effectively fatal to a deployed float.  Thorough laboratory
simulations of new configurations are strongly encouraged and careful
predeployment testing of each float is essential.

Each of the $\sim40$ tests is classified as either a constraint or a sanity
check.  Constraint violations are likely fatal to a deployed float and the
\cs\ will refuse to accept parameters or combinations of parameters that
violate a constraint.  Sanity checks detect various suspicious conditions
that are not likely fatal but that are probably inadvisable or unintended.

Each time the \cs\ encounters a violation, a verbal description of the
violation is given together with the C-source code for the test that was
violated.  The C-source code is expressed in terms of the mission
configuration parameters and can be used to figure out how to correct the
problem. 

\subsubsection{Missions impossible.}

Constraint violations represent missions that are not possible or else
potentially fatal to a deployed float.  The \cs\ will reject mission
configurations that violate constraints.  The following is a description of
each constraint:

\begin{description}
\begin{minipage}{6in}
\item[\textnormal{Range constraints are applied to most mission parameters:}]\ \\ 
  \begin{tabular}{llcccl} 
    & 0          & $\le$ & Verbosity                  & $\le$ & 5             \\
    & 1~Count    & $\le$ & AirBladderMaxP             & $\le$ & 240~Counts    \\
    & 1~Count    & $\le$ & OkVacuumCount              & $\le$ & 254~Counts    \\
    & 1~Count    & $\le$ & PistonBuoyancyNudge        & $\le$ & 254~Counts    \\
    & 1~Count    & $\le$ & DeepProfilePistonPos       & $\le$ & 254~Counts    \\
    & 1~Count    & $\le$ & PistonFullExtension        & $\le$ & 254~Counts    \\
    & 1~Count    & $\le$ & PistonInitialBuoyancyNudge & $\le$ & 254~Counts    \\
    & 0~Count    & $\le$ & PistonParkHyperRetraction  & $\le$ & 254~Counts    \\
    & 1~Count    & $\le$ & ParkPistonPos              & $\le$ & 254~Counts    \\
    & 1~Count    & $\le$ & PistonStoragePosition      & $\le$ & 254~Counts    \\
    & 1          & $\le$ & PnPCycleLen                & $\le$ & 254           \\
    & 0~Decibars & $\le$ & ParkPressure               & $\le$ & 2000~Decibars \\
    & 0~Decibars & $\le$ & DeepProfilePressure        & $\le$ & 2000~Decibars \\
    & 0~Minute   & $\le$ & DeepProfileDescentTime     & $\le$ & 8~Hours       \\
    & 0~Minute   & $<$   & DownTime                   & $\le$ & 30~Days       \\
    & 1~Minute   & $\le$ & AscentTimeOut              & $\le$ & 10~Hours      \\
    & 0~Minute   & $<$   & ParkDescentTime            & $\le$ & 8~Hours       \\
    & 0~Minute   & $<$   & TimePrelude                & $\le$ & 6~Hours       \\
    & 1~Minute   & $\le$ & TelemetryRetry             & $\le$ & 6~Hours       \\
    & 0~Minutes  & $<$   & UpTime                     & $\le$ & 24~Hours      \\
    & 30~Seconds & $\le$ & ConnectTimeOut             & $\le$ & 5~Minutes     \\
  \end{tabular}
\end{minipage}
  
\item[\textnormal{The up-time must allow for a deep profile plus 2 hours for telemetry:}]\ \\
  $mission.TimeUp >= (mission.PressureProfile/dPdt) + 2*Hour$
  
\item[\textnormal{The up-time must allow for a park profile plus 2 hours for telemetry:}]\ \\
  $mission.TimeUp >= (mission.PressurePark/dPdt) + 2*Hour$
  
\item[\textnormal{The up-time has to be greater than the ascent time-out period:}]\ \\
  $mission.TimeUp > mission.TimeOutAscent$

\item[\textnormal{The down-time has to be greater than the park-descent time plus deep-profile descent time:}]\ \\
  $mission.TimeDown > mission.TimeParkDescent+mission.TimeDeepProfileDescent$

\item[\textnormal{The profile pressure must be greater than (or equal to) the park pressure:}]\ \\
  $mission.PressureProfile >= mission.PressurePark$

\item[\textnormal{The primary dial command must begin with AT:}]\ \\
  !strncmp(mission.at, AT,2)

\item[\textnormal{The alternate dial command must begin with AT:}]\ \\
  !strncmp(mission.alt,AT,2)

\end{description}

\subsubsection{Missions insane.}

The \cs\ will warn the operator about violations of sanity checks but will not
reject the configuration.  The following is a list of each sanity check:

\begin{description}

\item[\textnormal{Ascent time should be sufficient for a deep profile:}]\ \\
  $mission.TimeOutAscent >= (mission.PressureProfile/dPdt) + 1*Hour$

\item[\textnormal{Ascent time should be sufficient for a park profile:}]\ \\
  $mission.TimeOutAscent >= (mission.PressurePark/dPdt) + 1*Hour$

\item[\textnormal{Up time should be sufficient to guarantee at least 2 hours for telemetry:}]\ \\
  $mission.TimeUp >= mission.TimeOutAscent + 2*Hour$

\item[\textnormal{Park descent period should be compatible with park pressure:}]\ \\
  $mission.TimeParkDescent >= mission.PressurePark/dPdt$

\item[\textnormal{Park descent period should not be excessive:}]\ \\
  $mission.TimeParkDescent <= 1.5*mission.PressurePark/dPdt + 1*Hour$

\item[\textnormal{Deep-profile descent period should be compatible with profile pressure:}]\ \\
  $mission.TimeDeepProfileDescent >= (mission.PressureProfile-mission.PressurePark)/dPdt)$

\item[\textnormal{Down time should be sufficient for active ballasting algorithm to adjust buoyancy:}]\ \\
  \mbox{$mission.TimeDown > mission.TimeDeepProfileDescent + mission.TimeParkDescent + 2*Hour$}

\item[\textnormal{Deep-profile descent period should not be excessive:}]\ \\
  \mbox{$mission.TimeDeepProfileDescent <= 1.5*(mission.PressureProfile-mission.PressurePark)/dPdt + Hour$}

\item[\textnormal{Profile piston position should be compatible with park piston position:}]\ \\
  $mission.PistonDeepProfilePosition <= mission.PistonParkPosition$
\item[\textnormal{Compensator instability: N2 float should be parked
    deeper.}]  This sanity check applies only if the compensator
    hyper-retraction is nonzero. \\
  $mission.PressurePark >= MinN2ParkPressure$


\item[\textnormal{Maximum air-bladder pressure seems insane:}]\ \\
  $120 \le mission.MaxAirBladder \le 128$

\item[\textnormal{The float serial number should be greater than zero:}]\ \\
  $mission.FloatId>0$

\end{description}


%%% Local Variables: 
%%% mode: latex
%%% TeX-master: "IridiumApex"
%%% End: 
