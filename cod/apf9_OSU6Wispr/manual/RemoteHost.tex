%~~~~~~~~~~~~~~~~~~~~~~~~~~~~~~~~~~~~~~~~~~~~~~~~~~~~~~~~~~~~~~~~~~~~~~~~
% $Id: RemoteHost.tex,v 1.1.2.1 2008/10/01 14:51:14 dbliudnikas Exp $
%~~~~~~~~~~~~~~~~~~~~~~~~~~~~~~~~~~~~~~~~~~~~~~~~~~~~~~~~~~~~~~~~~~~~~~~~
% RCS Log:
%
% $Log: RemoteHost.tex,v $
% Revision 1.1.2.1  2008/10/01 14:51:14  dbliudnikas
% Apf9i Seascan TD: Add Iridium serial number and testing.
%
% Revision 1.1  2006/11/03 19:08:57  swift
% Added user manual to CVS control.
%
% Revision 1.2  2006/07/10 22:24:49  swift
% Modifications to bring the manual up to date with
% changes to the SeaBird CTD firmware (v1.1c).
%
% Revision 1.1  2006/02/26 16:35:26  swift
% Initial revision
%
%~~~~~~~~~~~~~~~~~~~~~~~~~~~~~~~~~~~~~~~~~~~~~~~~~~~~~~~~~~~~~~~~~~~~~~~~

\newcommand{\mgetty}{\textbf{mgetty}}
\renewcommand{\-}{\~{\hspace{0in}}}

\section{The remote UNIX host.}
\label{sec:RemoteHost}  

This \iridium\ implementation uses a modem-to-modem communications model.  The
float initiates a telephone call to a remote host computer, logs into the
remote host with a username and password, executes a sequence of commands to
transfer data, and then logs out.  The communications session is float-driven

With respect to the remote host, there is no difference between the float
logging in and a human logging in.  The communications session is initiated
and fully controlled by the float.  On the other hand, the float is not
naturally adaptable or interactive like a human would be and so an unusual
amount of fault tolerance has been built into both sides of the
communications session.

An important fault tolerance measure is redundancy in the form of two
similarly configured remote hosts each with its own dedicated telephone
line.  This is optional but recommended.  Ideally, these two remote hosts
should be separated far enough from each other that power outages or
telephone outages are not likely to simultaneously affect both remote hosts.
The float firmware is designed to automatically switch to the alternate
remote host if with the primary remote host appears to be out of service.

\subsection{System requirements.}
\label{sec:SystemRequirements}

This \iridium\ implementation is strongly tied to the use of a UNIX computer
as the remote host (ie., Microsoft operating systems are not suitable).
\textbf{The most important ``system requirement'' is a system administrator
  that is familiar, comfortable, and competent in a UNIX environment.}
While many different flavors of UNIX could be made to work, development was
done using RedHat Linux (versions 7-9).  RedHat Linux (version 9) will be
assumed for the remainder of this section.

The \mgetty\ package must be installed and configured to monitor a
Hayes-compatible external modem attached to one of the serial ports.  For
information on how to install and configure the \mgetty\ package, refer to
the \mgetty\ documentation supplied with RedHat Linux.  If you customize the
login prompt, make sure that it includes the phrase ``login:''.  Similarly,
make sure that the password prompt includes the phrase ``Password:''.  The
float will not successfully log in if these two phrases are not present.

Once \mgetty\ is installed and configured properly, you should be able to log
into the remote host via a modem-to-modem connection from another computer.
You should test this using the following communications parameters:
4800baud, no parity, 8-bit data, 1 stop-bit.

\subsection{Remote host set up.} 
\label{sec:RemoteHostSetUp} 

Once each telemetry cycle, the float downloads ``mission.cfg'' from the home
directory where the float logs in and this new mission configuration becomes
active as the last step before the telemetry cycle terminates (see
Section~\ref{sec:RemoteControl}).  In the context of a UNIX environment,
this simple mechanism allows for great flexibility for remotely controlling
floats individually, in groups, or fleet-wise.  It is also flexible in that
it is possible to switch which model is used even after floats have been
deployed.  Finally, a UNIX-based remote host facilitates easy speciation of
floats as well as for new float developments with no requirement for
backward compatibility.

\subsubsection{Setting up the \emph{default user} on the remote host.}
\label{sec:RemoteHostDefaultUser}

Another fault tolerance measure requires creation of a \emph{default user}
on the remote host.  Begin by creating a new {\sl iridium}\/ group to which
the \emph{default user} and all floats will belong.  As root, execute the
command: 
\begin{quotation}
  {\sl groupadd -g1000 iridium}
\end{quotation}

Next, create an account for the \emph{default user}\/ using \emph{iridium}\/
as the username:
\begin{quotation}
  \mbox{\sl adduser~-s/bin/tcsh~-c"Iridium~Apex~Drifter"~-g"iridium"~-u1000~-d/home/iridium~iridium}
\end{quotation}
Then give the new user a password by executing (as root): 
\begin{quotation}
   {\sl passwd~iridium}
\end{quotation}
For the convenience of the float manager, you might also want to change the
permissions on the float's home directory: 
\begin{quotation}
   {\sl chmod~750~\-iridium}.
\end{quotation}
The file, {\sl /etc/passwd\/}, will contain the following entry:
\begin{verbatim}
   iridium:x:1000:1000:Iridium Apex Drifter:/home/iridium:/bin/tcsh
\end{verbatim}

The remainder of the set-up for this float should be done while logged into
the remote host as the \emph{default user}\/ (ie., \emph{iridium}\/).
Create two directories:
\begin{quotation}
   {\sl mkdir \-/bin \-/logs} 
\end{quotation}
and populate the {\sl \-/bin\/} directory with the \emph{SwiftWare}\/ xmodem
utilities \textbf{rx} and \textbf{sx} as well as the \textbf{chkconfig}
utility.  These three files are in the \textbf{support} directory of your
distribution.

Finally, use \textbf{emacs} to create the following three ascii files:
\textbf{.cshrc}, \textbf{.rxrc}, and \textbf{.sxrc}:
\begin{description}
\item[.cshrc:] This file configures the t-shell at login time.  You can
  modify the configuration to suit yourself so long as your customizations
  do not interfere with the effects that the three commands below have.  In
  particular, it is important that the float's \textbf{bin} directory be in
  the {\sl path\/} before any of the system directories.  This will ensure
  that the float's version of the utilities \textbf{chkconfig}, \textbf{rx},
  and \textbf{sx} will be used rather than the system's utilities with these
  same names.
  
  \begin{verbatim}
  # set the hostname
  set hostname=`hostname`
  
  # add directories for local commands
  set path = (. ~/bin /bin /sbin /usr/sbin /usr/local/bin)
  
  # set the prompt
  set prompt=""$hostname":[$cwd]> "
  \end{verbatim}
  
\item[.rxrc:] This is the configuration file for the \emph{SwiftWare}\/
  implementation the xmodem receive utility.  \emph{SwiftWare}\/ \textbf{rx}
  implements the standard xmodem protocol except that a nonstandard 16-bit
  CRC is used.  Beware that the float will not be able to transfer any
  hydrographic or engineering data to the remote host using the system
  version of \textbf{rx}.  Make sure that the {\sl LogPath}\/ references the
  \emph{default user's}\/ \textbf{logs} directory or else potentially
  valuable logging/debugging information will be irretrievably lost.

  \begin{verbatim}
  # This is the configuration file for 'rx', the 
  # SwiftWare xmodem receive utility.
  
  # set the default debug level (range: 0-4)
  Verbosity=5
  
  # specify the name of the log file
  LogPath=/home/iridium/logs/rxlog
  
  # enable (AutoLog!=0) or disable (AutoLog==0) the auto-log feature
  AutoLog=1
  
  # specify ascii mode (BinaryMode==0) or binary mode (BinaryMode!=0) 
  BinaryMode=0
  
  # specify CRC mode (16bit or 8bit)
  CrcMode=16bit
  \end{verbatim}
 
\item[.sxrc:] This is the configuration file for the \emph{SwiftWare}\/
  implementation the xmodem send utility.  \emph{SwiftWare}\/ \textbf{sx}
  implements the standard xmodem protocol except that a nonstandard 16-bit
  CRC is used.  Beware that new mission configurations will not be
  downloaded from the remote host to the float if system version of
  \textbf{sx} is used.  Make sure that the {\sl LogPath}\/ references the
  \emph{default user's}\/ \textbf{logs} directory or else potentially
  valuable logging/debugging information will be irretrievably lost.

  \begin{verbatim}
  # This is the configuration file for 'sx', the 
  # SwiftWare xmodem send utility.
  
  # set the default debug level (range: 0-4)
  Verbosity=5
  
  # specify the name of the log file
  LogPath=/home/iridium/logs/sxlog
  
  # enable (AutoLog!=0) or disable (AutoLog==0) the auto-log feature
  AutoLog=1
  
  # specify fixed packet type (128b or 1k)
  # PktType=1k
  \end{verbatim}

\end{description}

\subsubsection{Setting up the remote host for individualized remote control.}
\label{sec:RemoteHostUser} 

The ability to individualize each float is implemented by each float having
its own account on the remote host.  The steps to set up the remote host are
analagous to those for setting up the \emph{default user}\/ (see
Section~\ref{sec:RemoteHostDefaultUser}).  For example, to create an account
for float~5047 then make sure the \textbf{iridium} group exists (see
Section~\ref{sec:RemoteHostDefaultUser}) and then execute the following
command (as root):
\begin{quotation}
  \mbox{\sl adduser~-s/bin/tcsh~-c"Iridium~Apex~Drifter"~-g"iridium"~-u15047~-d/home/f5047~f5047}
\end{quotation}
Then give the new user a password and change the permissions of the float's
home directory as shown for the \emph{default user}.  \textbf{Be sure to
  configure the float to use this username and password} (see
Section~\ref{sec:MissionConfiguration}).  The file, \textbf{/etc/passwd},
will contain the following entry:
\begin{verbatim}
   f5047:x:15047:1000:Iridium Apex Drifter:/home/f5047:/bin/tcsh
\end{verbatim}

The remainder of the set-up for this float follows very closely that of the
\emph{default user}\/ and should be done while logged into the remote host
as the float (ie., \emph{f5047}\/).  Create \textbf{bin} and \textbf{logs}
directories in the float's home directory and populate the \textbf{bin}
directory with the \emph{SwiftWare}\/ xmodem utilities \textbf{rx} and
\textbf{sx} as well as the \textbf{chkconfig} utility.

Finally, copy the three ascii files \textbf{.cshrc}, \textbf{.rxrc}, and
\textbf{.sxrc} from the \emph{default user's}\/ home directory to the
float's home directory.  Be sure to edit these files so that the {\sl
  LogPath}\/ points to the float's \textbf{logs} directory or else
potentially valuable logging/debugging information will be irretrievably
lost.

\subsubsection{Setting up the remote host for fleet-wise remote control.} 
\label{sec:RemoteHostFleet} 

The flexibility inherent with individualized float control necessarily
increases the level of operational management required---each float has to
be considered and controlled individually.  However, fleet-wise management
of floats is also made possible by configuring the float to use a fleet-wise
username.  This is in contrast to Section~\ref{sec:RemoteHostUser} where
each float was configured with a unique username (based on the float serial
number).  The steps to set up the remote host for fleet-wise control are
virtually the same as those in
Sections~\ref{sec:RemoteHostDefaultUser}~\&~\ref{sec:RemoteHostUser} except
that the username and password are fleet-wise parameters.  Be sure to
configure each float in the fleet with the fleet-wise username and password.


\subsubsection{Iridium Serial Number Handling} 

The SIM phone book is programmed with the phone number assigned to the SIM (MSISDN)
Mobile System Integrated Services Digital Network and the SIM ID number (ICCID)
Integrated Circuit Card Identification.  The firmware will be able to query the
LBT because it canot be done through the standard AT commands.  This protects
against SIM cards betgin swapped between floats.  If a float fails afer deployment,
there is a record of twhich SIM card is coupled to a particular float that allows
the SIM card to be deactivated.

This information can be programmed using a terminal emulation program (19200 N-8-1),
LBT test fixture and the following commands:
 
\begin{verbatim}
1. AT (wait 60 seconds)
2. AT+CPBS="SM"
3. AT+CPBW=101,"<CCID>",129,"<MSISDN>"  where <CCID> and <MSISDN> are actual values
4. AT+CPBS? 
5. AT+CPBR=101 (to verify)

Note: If the PIN needs to be set, use the following command: AT+CPIN="<PIN>".
\end{verbatim}

\subsubsection{GPS, LBT, Remote Host Testing}

After connecting to an iridium antenna, enter sail commands via a terminal emulation
program (9600 8-N-1).  

\begin{verbatim}
Press any key to enter command mode ">" prompt.

1. gc (configure the GPS)
2. gp (access satellites and set the real time clock)
3. gf (get a GPS fix)
4. hc (configure the Iridium modem)
5. r (activate recovery mode where a GPS fix is followed by an Iridum call and 
      once connection established, upload mission.cfg and 
      download log information to the remote host.)
6. k (when complete, kill mission by responding with a "y" acknowledgement)
\end{verbatim}

%%% Local Variables: 
%%% mode: latex
%%% TeX-master: "IridiumApex"
%%% End: 
