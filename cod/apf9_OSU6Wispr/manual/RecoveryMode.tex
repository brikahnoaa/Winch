%~~~~~~~~~~~~~~~~~~~~~~~~~~~~~~~~~~~~~~~~~~~~~~~~~~~~~~~~~~~~~~~~~~~~~~~~
% $Id: RecoveryMode.tex,v 1.1 2006/11/03 19:08:57 swift Exp $
%~~~~~~~~~~~~~~~~~~~~~~~~~~~~~~~~~~~~~~~~~~~~~~~~~~~~~~~~~~~~~~~~~~~~~~~~
% RCS Log:
%
% $Log: RecoveryMode.tex,v $
% Revision 1.1  2006/11/03 19:08:57  swift
% Added user manual to CVS control.
%
% Revision 1.2  2006/07/10 22:24:49  swift
% Modifications to bring the manual up to date with
% changes to the SeaBird CTD firmware (v1.1c).
%
% Revision 1.1  2006/04/11 14:48:16  swift
% Initial revision
%
%~~~~~~~~~~~~~~~~~~~~~~~~~~~~~~~~~~~~~~~~~~~~~~~~~~~~~~~~~~~~~~~~~~~~~~~~

\section{Recovery mode.}
\label{sec:RecoveryMode}

As its name suggests, recovery mode is intended primarily to facilitate
post-deployment recovery of the float from the ocean.  However, its
operational behaviors are general enough to allow for many other useful
applications, too. Recovery mode is fundamentally a remote control feature.
Section~\ref{sec:RemoteControl} describes the facilility for remote control
of \iridium\ floats using 2-way commands.  The
\textbf{ActivateRecoveryMode()} command is used to both initiate and
maintain recovery mode for as long as the operator desires.

The recovery mode cycle operates on the telemetry-retry period.  Each cycle
starts by ensuring that the piston is fully extended and the air bladder is
fully inflated.  Then a GPS fix is obtained and telemetry is initiated to
upload the GPS fix and a small amount of engineering data to the remote
host.  The pathname for the file has the pattern \emph{FloatId.YYMMDDhhmm}
where \emph{FloatId} is the 4-digit float identifier and \emph{YYMMDDhhmm}
represents the date \& time when the recovery cycle was initiated.  It is a
simple matter to arrange for the remote host to automatically relay the GPS
fix to an \iridium\ handset on-board the ship.  This allows recovery
operations to be conducted without on-shore aid.

The new mission configuration file will also be downloaded from the remote
host.  If the new mission configuration file contains the
\textbf{ActivateRecoveryMode()} command then the float will go to sleep for
one telemetry-retry period and then wake up to repeat the recovery mode
cycle.  If the new mission configuration file does not contain the
\textbf{ActivateRecoveryMode()} command then subsequent float behavior
depends upon what the float was doing before recovery mode was initiated.
If the float was in its mission prelude then the mission prelude is
re-initiated\footnote{The mission prelude is not merely continued where it
  left off when recovery mode was activated; the mission prelude is
  completely restarted}.

On the other hand, if the float's mission was in progress when recovery mode
was initiated then upon termination of recovery mode the mission resumes
where it left off.  That is, if profile~\emph{N\/} had been telemetered just
prior to initiation of recovery mode then the descent phase of profile
cycle~\emph{N+1\/} will begin as soon as recovery mode is terminated.

%%% Local Variables: 
%%% mode: latex
%%% TeX-master: "IridiumApex"
%%% End: 
